\documentclass[reqno,a4paper,12pt]{amsart}

\usepackage{amsmath,amssymb,amsthm,geometry,xcolor,soul,graphicx}
\usepackage{titlesec}
\usepackage{enumerate}
\usepackage{lipsum}
\usepackage{listings}
%\RequirePackage[most]{tcolorbox}
\usepackage{braket}
\allowdisplaybreaks[4] %align公式跨页
\usepackage{xeCJK}
\setCJKmainfont[AutoFakeBold = true]{Kai}
\geometry{left=0.7in, right=0.7in, top=1in, bottom=1in}

\renewcommand{\baselinestretch}{1.3}

\title{介观物理第十次作业}
\author{董建宇 ~~ 202328000807038}

\begin{document}

\maketitle
\titleformat{\section}[hang]{\small}{\thesection}{0.8em}{}{}
\titleformat{\subsection}[hang]{\small}{\thesubsection}{0.8em}{}{}

\textbf{Problem III.6} Verify that 
\[
	1 - R_{ii} - \sum_{j\neq i} T_{ij} = 0
\]

for the derivation of Eq. (210).

\begin{proof}

由于时间反演对称性,$S$矩阵满足幺正性。此时$S$矩阵可以写作:
\[
	S = \left( \begin{matrix}
		r & t \\
		t & r
	\end{matrix} \right).
\]

结合上一次作业得到的结论,有:
\[
	R_i + T_i = 1.
\]

在这种情况下,$R_{ij} = \delta_{ij}$,$T_{ii} = 0$,则上式可重写为:
\[
	R_{ii} + \sum_{j\neq i} T_{ij} = 1.
\]

即:
\[
	1 - R_{ii} - \sum_{j\neq i}T_{ij} = 0.
\]

从物理图像理解,该式表达了透射概率和反射概率和为一。

\end{proof}

\end{document}