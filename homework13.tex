\documentclass[reqno,a4paper,12pt]{amsart}

\usepackage{amsmath,amssymb,amsthm,geometry,xcolor,soul,graphicx}
\usepackage{titlesec}
\usepackage{enumerate}
\usepackage{lipsum}
\usepackage{listings}
%\RequirePackage[most]{tcolorbox}
\usepackage{braket}
\usepackage{dutchcal} %could use \mathcal{} format for minuscule
%\usepackage{mathrsfs} %could use \mathscr{} format
\allowdisplaybreaks[4] %align公式跨页
\usepackage{xeCJK}
\setCJKmainfont[AutoFakeBold = true]{Kai}
\geometry{left=0.7in, right=0.7in, top=1in, bottom=1in}

\renewcommand{\baselinestretch}{1.3}

\title{介观物理第十三次作业}
\author{董建宇 ~~ 202328000807038}

\begin{document}

\maketitle
%\titleformat{\section}[hang]{\small}{\thesection}{0.8em}{}{}
%\titleformat{\subsection}[hang]{\small}{\thesubsection}{0.8em}{}{}

电子在周期势场$V(\mathbf{r})$中运动,满足薛定谔方程为:
\[
	H \ket{\psi_{n\mathbf{k}}} = \varepsilon \ket{\psi_{n\mathbf{k}}},
\]

其中
\[
	H = \frac{1}{2m} [-i\hbar\nabla + e\mathbf{A}(\mathbf{r})]^2 + V(\mathbf{r}),
\]

其中$\mathbf{A}(\mathbf{r})$为电磁矢势。假设$\ket{\psi_{n\mathbf{k}}} = e^{i\mathbf{k}\cdot\mathbf{r}} \ket{u_{n\mathbf{k}}}$满足:
\[
	\mathcal{h} \ket{u_{n\mathbf{k}}} = \varepsilon_{n\mathbf{k}} \ket{u_{n\mathbf{k}}}.
\]

\begin{enumerate}

\item 请证明:
\[
	\mathcal{h} = \frac{1}{2m}[-i\hbar\nabla + \hbar\mathbf{k} + e\mathbf{A}(\mathbf{r})]^2 + V(\mathbf{r}),
\]

\begin{proof}

将哈密顿量展开,写作:
\[
	H = \frac{1}{2m}[-i\hbar\nabla + e\mathbf{A}]^2 + V = \frac{1}{2m}(-\hbar^2\nabla^2 + e^2\mathbf{A}^2 - 2ie\hbar\mathbf{A}\cdot\nabla) + V.
\]

分别作用在$\ket{\psi_{n\mathbf{k}}} = e^{i\mathbf{k} \cdot \mathbf{r}} \ket{u_{n\mathbf{k}}}$可得:
\begin{align*}
	\nabla \cdot \nabla(e^{i\mathbf{k} \cdot \mathbf{r}} \ket{u_{n\mathbf{k}}}) =& \nabla \cdot \left( e^{i\mathbf{k} \cdot \mathbf{r}} \nabla \ket{u_{n\mathbf{k}}} + i\mathbf{k} e^{i\mathbf{k} \cdot \mathbf{r}} \ket{u_{n\mathbf{k}}} \right) \\
	=& e^{i\mathbf{k} \cdot \mathbf{r}} \nabla^2 \ket{u_{n\mathbf{k}}} + 2i\mathbf{k} e^{i\mathbf{k} \cdot \mathbf{r}} \cdot \nabla \ket{u_{n\mathbf{k}}} - \mathbf{k}^2 e^{i\mathbf{k} \cdot \mathbf{r}} \ket{u_{n\mathbf{k}}} \\
	=& e^{i\mathbf{k} \cdot \mathbf{r}} (\nabla + i\mathbf{k})^2 \ket{u_{n\mathbf{k}}}; \\
	\nabla \left( e^{i\mathbf{k} \cdot \mathbf{r}} \ket{u_{n\mathbf{k}}} \right) =& e^{i\mathbf{k} \cdot \mathbf{r}} (\nabla+i\mathbf{k}) \ket{u_{n\mathbf{k}}}.
\end{align*}

从而可以计算:
\begin{align*}
	H \ket{\psi_{n\mathbf{k}}} =& e^{i\mathbf{k} \cdot \mathbf{r}} \left[ \frac{1}{2m}((-i\hbar\nabla + \hbar\mathbf{k})^2 + 2e\mathbf{A} \cdot (-i\hbar\nabla + \hbar\mathbf{k}) + e^2\mathbf{A}^2) + V \right] \ket{u_{n\mathbf{k}}} \\
	=& e^{i\mathbf{k} \cdot \mathbf{r}} \left[ \frac{1}{2m} (-i\hbar\nabla + \hbar\mathbf{k} + e\mathbf{A})^2 + V \right] \ket{u_{n\mathbf{k}}} \\
	=& e^{i\mathbf{k} \cdot \mathbf{r}} \varepsilon_{n\mathbf{k}} \ket{u_{\mathbf{k}}}.
\end{align*}

即有:
\[
	\mathcal{h} = \frac{1}{2m}[-i\hbar\nabla + \hbar\mathbf{k} + e\mathbf{A}(\mathbf{r})]^2 + V(\mathbf{r}).
\]
\end{proof}


\item 速度算符可以写成:
\[
	\mathbf{v} = \frac{1}{m}[-i\hbar\nabla + e\mathbf{A}(\mathbf{r}) + \hbar\mathbf{k}] = \frac{1}{\hbar}\frac{\partial \mathcal{h}}{\partial \mathbf{k}}.
\]

\begin{proof}
注意到:$\mathcal{h}$可以写成动能项和势能项的和,即:
\[
	\mathcal{h} = \frac{1}{2}m \mathbf{v}^2 + V(\mathbf{r}).
\]

从而不难得到:
\[
	\mathbf{v} = \frac{1}{m}[-i\hbar\nabla + e\mathbf{A}(\mathbf{r}) + \hbar\mathbf{k}] = \frac{1}{\hbar}\frac{\partial \mathcal{h}}{\partial \mathbf{k}}.
\]
\end{proof}


\item 证明:
\[
	\bra{u_{n'\mathbf{k}}} [\mathbf{r}, \mathcal{h}] \ket{u_{n\mathbf{k}}} = -i\hbar \bra{u_{n'\mathbf{k}}} \mathbf{v} \ket{u_{n\mathbf{k}}}.
\]

\begin{proof}
要证明上式,只需计算$\mathbf{r}$和$\mathcal{h}$的对易关系,具体计算如下:
\begin{align*}
	[\mathbf{r}, \mathcal{h}] =& \frac{1}{2m}[\mathbf{r}, (-i\hbar\nabla + \hbar\mathbf{k} + e\mathbf{A})^2] \\
	=& \frac{1}{2m} [\mathbf{r}, -\hbar^2\nabla^2 + (\hbar\mathbf{k} + e\mathbf{A})^2 - 2i\hbar(\hbar\mathbf{k} + e\mathbf{A}) \cdot \nabla] \\
	=& \frac{1}{2m}[\mathbf{r}, \mathbf{p}^2] + \frac{1}{m}[\mathbf{r}, (\hbar\mathbf{k}+e\mathbf{A}) \cdot \mathbf{p}] \\
	=& \frac{i\hbar}{m} \mathbf{p} + \frac{1}{m} (\hbar\mathbf{k} + e\mathbf{A}) i\hbar \\
	=& i\hbar \frac{1}{m}(-i\hbar\nabla + \hbar\mathbf{k} + e\mathbf{A}) \\
	=& i\hbar \mathbf{v}.
\end{align*}

其中,$\mathbf{p} = -i\hbar\nabla$,此外利用了基本对易关系:
\[
	[x_i, p_j] = i\hbar\delta_{ij}, \ \ i,j = 1,2,3.
\]

即有:
\[
	\bra{u_{n'\mathbf{k}}} [\mathbf{r}, \mathcal{h}] \ket{u_{n\mathbf{k}}} = i\hbar \bra{u_{n'\mathbf{k}}} \mathbf{v} \ket{u_{n\mathbf{k}}}.
\]
\end{proof}

\end{enumerate}





\end{document}