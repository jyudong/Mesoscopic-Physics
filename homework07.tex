\documentclass[reqno,a4paper,12pt]{amsart}

\usepackage{amsmath,amssymb,amsthm,geometry,xcolor,soul,graphicx}
\usepackage{titlesec}
\usepackage{enumerate}
\usepackage{lipsum}
\usepackage{listings}
%\RequirePackage[most]{tcolorbox}
\usepackage{braket}
\allowdisplaybreaks[4] %align公式跨页
\usepackage{xeCJK}
\setCJKmainfont[AutoFakeBold = true]{Kai}
\geometry{left=0.7in, right=0.7in, top=1in, bottom=1in}

\renewcommand{\baselinestretch}{1.3}

\title{介观物理第七次作业}
\author{董建宇 ~~ 202328000807038}

\begin{document}

\maketitle
\titleformat{\section}[hang]{\small}{\thesection}{0.8em}{}{}
\titleformat{\subsection}[hang]{\small}{\thesubsection}{0.8em}{}{}

\textbf{Problem II.5} Assuming $g(L_0) = g_0$, where $L_0$ is a certain microscopic length scale, find the solution $g(L)$ to the scaling equation (148) at $d=2$ by using Eq. (154). The results can be found in Eq. (167).
\[
	\frac{d\ln g}{d \ln L} = \beta[g(L)]. \tag{148}
\]

\[
	\beta(g) = d-2-\frac{a}{g}. \tag{154}
\]

\[
	g(L) = g_0 - \frac{1}{\pi^2} \ln \left( \frac{L}{L_0} \right). \tag{167}
\]

\begin{proof}
方程(148)可以写作:
\[
	\frac{d\ln g}{d\ln L} = \frac{L}{d} \frac{dg}{dL} = \beta[g(L)].
\]

其中,利用了$d\ln g = \frac{1}{g}dg$。当$d=2$时,$\beta(g) = -\frac{a}{g}$。即有:
\[
	-\frac{dg}{a} = \frac{dL}{L}.
\]

两侧不定积分,可得:
\[
	-\frac{g}{a} = \ln L + C.
\]

其中$C$为常数。当$L=L_0$时,有:
\[
	-\frac{g_0}{a} = \ln L_0 + C.
\]

从而确定常数$C$为
\[
	C = -\frac{g_0}{a} - \ln L_0.
\]

从而有:
\[
	g(L) = g_0 - a\ln \left( \frac{L}{L_0} \right) = g_0 - \frac{1}{\pi^2} \ln \left( \frac{L}{L_0} \right).
\]

利用了$a = \frac{1}{\pi^2}$。
\end{proof}

\textbf{Problem II.6} Find the solution in Eq. (170).
\[
	\sigma(L) = g_0 L_0 - a(L-L_0). \tag{170}
\]

\begin{proof}
当$d=1$时,$\beta(g) = -1-\frac{a}{g}$,即方程(148)可化为:
\[
	-\frac{dg}{g+a} = \frac{dL}{L}.
\]

两侧不定积分,可得:
\[
	-\ln(g+a) = \ln L + C_2.
\]

其中,$C_2$为积分常数,当$L=L_0$时,有:
\[
	-\ln(g_0+a) = \ln L_0 + C_2.
\]

从而确定积分常数$C_2$为:
\[
	C_2 = -\ln(g_0+a) - \ln L_0.
\]

则$d=1$时,微分方程的解为:
\[
	\ln \left( \frac{g+a}{g_0+a} \right) = \ln \left( \frac{L_0}{L} \right).
\]

两侧取$e$指数,可得:
\[
	\frac{g+a}{g_0+a} = \frac{L_0}{L}.
\]

即:
\[
	g = \frac{L_0}{L}(g_0+a) - a.
\]

则电导率为:
\[
	\sigma(L) = gL = g_0L_0 - a(L-L_0).
\]
\end{proof}


\end{document}