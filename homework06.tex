\documentclass[reqno,a4paper,12pt]{amsart}

\usepackage{amsmath,amssymb,amsthm,geometry,xcolor,soul,graphicx}
\usepackage{titlesec}
\usepackage{enumerate}
\usepackage{lipsum}
\usepackage{listings}
%\RequirePackage[most]{tcolorbox}
\usepackage{braket}
\allowdisplaybreaks[4] %align公式跨页
\usepackage{xeCJK}
\setCJKmainfont[AutoFakeBold = true]{Kai}
\geometry{left=0.7in, right=0.7in, top=1in, bottom=1in}

\renewcommand{\baselinestretch}{1.3}

\title{介观物理第六次作业}
\author{董建宇 ~~ 202328000807038}

\begin{document}

\maketitle
\titleformat{\section}[hang]{\small}{\thesection}{0.8em}{}{}
\titleformat{\subsection}[hang]{\small}{\thesubsection}{0.8em}{}{}


\textbf{Problem II.3} Follow the above argument, derive the relation $G \propto e^{-(T_0/T)^{1/(d+1)}}$ in $d$ dimensions.

\begin{proof}

在充分低温度下,即$k_BT << \vert \epsilon_i \vert, \vert \epsilon_j \vert, \vert \epsilon_i - \epsilon_j \vert$,我们可以把电流表达式展开到$eV$一阶项,即:
\[
	I = \frac{eR\gamma_0}{k_BT} e^{-W/k_BT} e^{-2R/\xi} (eV).
\]

即电导为:
\[
	G = \frac{e^2R\gamma_0}{k_BT} \left[ e^{-W/k_BT} e^{-2R/\xi} \right].
\]

对于维度为$d$的体系有:
\[
	WAR^d = \frac{1}{N(0)}.
\]

其中$A$为常数,$d=3$时,$A = \frac{4\pi}{3}$。则可以计算:
\[
	\frac{dW}{dR} = \frac{d}{dR} \frac{1}{AN(0)R^d} = -\frac{d}{AN(0)}\frac{1}{R^{d+1}}.
\]

最大化电导表达式中括号内的项,令
\[
	f(R) = e^{-W/k_BT} e^{-2R/\xi}.
\]

可以计算:
\[
	\frac{df(R)}{dR} = e^{-W/k_BT} e^{-2R/\xi} \left( \frac{d}{Ak_BTN(0)} \frac{1}{R^{d+1}} - \frac{2}{\xi} \right).
\]

当$\frac{df(R)}{dR} = 0$时,有:
\[
	R_0^{d+1} = \frac{d\xi}{2Ak_BTN(0)}.
\]

当$R< R_0$时,$\frac{df(R)}{dR}>0$;当$R>R_0$时,$\frac{df(R)}{dR}<0$。即当$R = R_0$时,电导最大,此时有:
\[
	G \propto e^{-(2^dR^d/\xi^d)^{1/(d+1)}} = \exp \left[ -\left( \frac{d2^d}{Ak_BT N(0)\xi^d} \right)^{1/(d+1)} \right]
\]

其中$N(0)\xi^d = \frac{1}{k_BT_0}$,则有:
\[
	G \propto \exp \left[ -\left( \frac{d2^d}{A} \frac{T_0}{T} \right)^{1/(d+1)} \right]
\]

\end{proof}


\end{document}