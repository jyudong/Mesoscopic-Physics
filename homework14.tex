\documentclass[reqno,a4paper,12pt]{amsart}

\usepackage{amsmath,amssymb,amsthm,geometry,xcolor,soul,graphicx}
\usepackage{titlesec}
\usepackage{enumerate}
\usepackage{lipsum}
\usepackage{listings}
%\RequirePackage[most]{tcolorbox}
\usepackage{braket}
\usepackage{dutchcal} %could use \mathcal{} format for minuscule
%\usepackage{mathrsfs} %could use \mathscr{} format
\allowdisplaybreaks[4] %align公式跨页
\usepackage{xeCJK}
\setCJKmainfont[AutoFakeBold = true]{Kai}
\geometry{left=0.7in, right=0.7in, top=1in, bottom=1in}

\renewcommand{\baselinestretch}{1.3}

\title{介观物理第十四次作业}
\author{董建宇 ~~ 202328000807038}

\begin{document}

\maketitle
%\titleformat{\section}[hang]{\small}{\thesection}{0.8em}{}{}
%\titleformat{\subsection}[hang]{\small}{\thesubsection}{0.8em}{}{}

\begin{enumerate}[1.]

\item 在$\mathbf{K}$和$\mathbf{K}'$点附近做小量近似,计算Haldane模型相图中$C=1$和$C=0$两相边界的表达式。

\begin{proof}

Haldane在文章中写出$M \neq 0, \ \cos\phi\neq 0, \ \sin\phi\neq 1$的哈密顿量为:
\begin{align*}
	\hat{H} =& 2t_2\cos\phi \left[ \sum_i \cos(\vec{k}\cdot\vec{b}_i) \right] \mathbf{I} + t_1 \left[ \sum_i \cos(\vec{k}\cdot\vec{a}_i) \sigma^1 + \sum_i \sin(\vec{k}\cdot\vec{a}_i) \sigma^2 \right] \\
	&+ \left[ M - 2t_2 \sin\phi \sum_i \sin(\vec{k}\cdot\vec{b}_i) \right]\sigma^3
	%\hbar c (\tau_z \sigma_x q_x + \sigma_y q_y) + M \sigma_z + \left( M - \frac{3\sqrt{3}}{2} t_2 \sin\phi \right)\tau_z\sigma_z.
\end{align*}

其中对于$\mathbf{K}$和$\mathbf{K}'$点,有:
\[
	\vec{k} \cdot \vec{b}_i = \frac{2\pi}{3}; \ \ \vec{k} \cdot \vec{b}_i = -\frac{2\pi}{3}.
\]

Haldane模型相图中$C=1$和$C=0$的边界条件为$\sigma^3$项系数为$0$,即有:
\[
	M - 2t_2 \sin\phi \sum_i \frac{\sqrt{3}}{2} = 0.
\]

即有:
\[
	\frac{M}{t_2} = 3\sqrt{3}\sin\phi.
\]

类似的,$C=-1$和$C=0$的边界条件为:
\[
	M - 2t_2\sin\phi \sum_i \left( -\frac{\sqrt{3}}{2} \right) = 0.
\]

即
\[
	\frac{M}{t_2} = -3\sqrt{3}\sin\phi.
\]
\end{proof}


\item 考虑哈密顿量$\hat{H}(\mathbf{k}) = \mathbf{\sigma} \cdot \mathbf{k}$,其中$\mathbf{\sigma} = \sigma_x\mathbf{e}_x + \sigma_y\mathbf{e}_y + \sigma_z\mathbf{e}_z$,$\mathbf{k}$为波矢,$\sigma_i$为泡利矩阵。请利用$\mathbf{A} = i\bra{\psi^-} \nabla_k \ket{\psi^-}$以及$\mathbf{b} = \nabla \times \mathbf{A}$求解本征解$E = -\vert \mathbf{k} \vert$对应的贝里场$\mathbf{b}$。

\begin{proof}

根据题意,哈密顿量可以写作:
\[
	H(\mathbf{k}) = \mathbf{\sigma} \cdot \mathbf{k} = \left( \begin{matrix}
		k_z & k_x - ik_y \\
		k_x + ik_y & k_z
	\end{matrix} \right)
\]

本征值满足:
\[
	\left( \begin{matrix}
		k_z - \lambda & k_x-ik_y \\
		k_x+ik_y & -k_z - \lambda
	\end{matrix} \right) = \lambda^2 - \vert \mathbf{k} \vert^2 = 0.
\]

即两个本征能量为:$\lambda = \pm \vert \mathbf{k} \vert$。当本征能量为$\lambda = -\vert \mathbf{k} \vert$,本征矢满足:
\[
	\left( \begin{matrix}
		k_z & k_x-ik_y \\
		k_x+ik_y & -k_z
	\end{matrix} \right) 
	\left( \begin{matrix}
		a \\
		b
	\end{matrix} \right) = -\vert \mathbf{k} \vert 
	\left( \begin{matrix}
		a \\
		b
	\end{matrix} \right)
\]

即有:
\[
	a = \frac{k_z-\vert \mathbf{k} \vert}{k_x+ik_y} b.
\]

结合归一化条件$\vert a \vert^2 + \vert b \vert^2 = 1$,可得:
\[
	a = \frac{k_z-\vert \mathbf{k} \vert}{k_x+ik_y}\sqrt{\frac{\vert \mathbf{k} \vert + k_z}{2\vert \mathbf{k} \vert}}; \ \ b = \sqrt{\frac{\vert \mathbf{k} \vert + k_z}{2\vert \mathbf{k} \vert}}.
\]

则贝里联络为:
\begin{align*}
	\mathbf{A} =& i \left( \begin{matrix}
		a^* & b^*
	\end{matrix} \right) \nabla_k \left( \begin{matrix}
		a \\
		b
	\end{matrix} \right) \\
	=& i\left(a^* \frac{\partial a}{\partial k_x} + b^* \frac{\partial b}{\partial k_x} \right) \mathbf{e}_x + i\left(a^* \frac{\partial a}{\partial k_y} + b^* \frac{\partial b}{\partial k_y} \right) \mathbf{e}_y + i\left(a^* \frac{\partial a}{\partial k_z} + b^* \frac{\partial b}{\partial k_z} \right) \mathbf{e}_z.
\end{align*}

为了简化书写,接下来使用$k$替代$\vert \mathbf{k} \vert$。接下来分别计算各项如下:
\begin{align*}
	\frac{\partial a}{\partial k_x} =&-\frac{1}{k_x+ik_y} \sqrt{\frac{k+k_z}{2k}} \frac{\partial k}{\partial k_x} - \frac{k_z-k}{k_x+ik_y}\frac{k_z}{4k^2}\sqrt{\frac{2k}{k+k_z}} \frac{\partial k}{\partial k_x} - i\frac{k_z-k}{(k_x+ik_y)^2}\sqrt{\frac{k+k_z}{2k}}; \\
	\frac{\partial b}{\partial k_x} =& -\sqrt{\frac{2k}{k+k_z}} \frac{k_z}{4k^2} \frac{\partial k}{\partial k_x}; \\
	\frac{\partial a}{\partial k_y} =& -\frac{1}{k_x+ik_y} \sqrt{\frac{k+k_z}{2k}} \frac{\partial k}{\partial k_y} - \frac{k_z-k}{k_x+ik_y}\frac{k_z}{4k^2}\sqrt{\frac{2k}{k+k_z}} \frac{\partial k}{\partial k_y} - i\frac{k_z-k}{(k_x+ik_y)^2}\sqrt{\frac{k+k_z}{2k}};\\
	\frac{\partial b}{\partial k_y} =& -\sqrt{\frac{2k}{k+k_z}} \frac{k_z}{4k^2} \frac{\partial k}{\partial k_y};\\
	\frac{\partial a}{\partial k_z} =& \frac{1-\frac{\partial k}{\partial k_z}}{k_x+ik_y} \sqrt{\frac{k+k_z}{2k}} + \frac{k_z-k}{k_x+ik_y} \sqrt{\frac{2k}{k+k_z}} \frac{1}{4k^2} \left( k-k_z\frac{\partial k}{\partial k_z} \right);\\
	\frac{\partial b}{\partial k_z} =& \sqrt{\frac{2k}{k+k_z}} \frac{1}{4k^2} \left( k-k_z\frac{\partial k}{\partial k_z} \right).
\end{align*}

从而可以计算:
\begin{align*}
	a^* \frac{\partial a}{\partial k_x} + b^* \frac{\partial b}{\partial k_x} =& \frac{k-k_z}{2k} \frac{ik_y}{k_x^2+k_y^2}; \\
	a^* \frac{\partial a}{\partial k_y} + b^* \frac{\partial b}{\partial k_y} =& -\frac{ik_x(k-k_z)}{2k(k_x^2+k_y^2)}; \\
	a^* \frac{\partial a}{\partial k_z} + b^* \frac{\partial b}{\partial k_z} =& 0.
\end{align*}

即贝里联络为:
\[
	\mathbf{A} = \frac{k_y(k_z-k)}{2k(k_x^2+k_y^2)} \mathbf{e}_x + \frac{k_x(k-k_z)}{2k(k_x^2+k_y^2)} \mathbf{e}_y.
\]

贝里场为:
\[
	\mathbf{b} = \nabla \times \mathbf{A} = -\frac{\partial A_y}{\partial k_z} \mathbf{e}_x + \frac{\partial A_x}{\partial k_z} \mathbf{e}_y + \left( \frac{\partial A_y}{\partial k_x} - \frac{\partial A_x}{\partial k_y} \right) \mathbf{e}_z
\]

可以计算:
\begin{align*}
	-\frac{\partial A_y}{\partial k_z} =& \frac{k_x}{2k^3}; \\
	\frac{\partial A_x}{\partial k_z} =& \frac{k_y}{2k^3}; \\
	\frac{\partial A_y}{\partial k_x} - \frac{\partial A_x}{\partial k_y} =& \frac{k_z}{2k^3}.
\end{align*}

即贝里场为:
\[
	\mathbf{b} = \frac{\mathbf{k}}{2k^3}.
\]

\end{proof}



\end{enumerate}

\end{document}