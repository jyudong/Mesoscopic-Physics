\documentclass[reqno,a4paper,12pt]{amsart}

\usepackage{amsmath,amssymb,amsthm,geometry,xcolor,soul,graphicx}
\usepackage{titlesec}
\usepackage{enumerate}
\usepackage{lipsum}
\usepackage{listings}
%\RequirePackage[most]{tcolorbox}
\usepackage{braket}
\allowdisplaybreaks[4] %align公式跨页
\usepackage{xeCJK}
\setCJKmainfont[AutoFakeBold = true]{Kai}
\geometry{left=0.7in, right=0.7in, top=1in, bottom=1in}

\renewcommand{\baselinestretch}{1.3}

\title{介观物理第五次作业}
\author{董建宇 ~~ 202328000807038}

\begin{document}

\maketitle
\titleformat{\section}[hang]{\small}{\thesection}{0.8em}{}{}
\titleformat{\subsection}[hang]{\small}{\thesubsection}{0.8em}{}{}

\textbf{Problem II.1} Read and Review: Read Anderson's 1958 paper entitled "Absence of Diffusion in Certain Random Lattice". Write a short review of this paper, no less than 1000 words or 350 Chinese characters.

\textsc{review:}这篇由$P. \ W. \ Anderson$于1958年发表的《Absence of diffusion in certain random lattices》文章,阐述了一个简单的随机格点模型,用于描述量子力学过程中的扩散现象,如自旋扩散或杂质带导电。主要内容可以概括如下:

首先,$Anderson$提出一个简单的理论模型,设置分布在三维空间的格点,且每个格点上的能量$E_i$是随机分布的,近邻格点之间存在相互作用矩阵元$V_{jk}$使能够扩散的载体(如自旋或电子)在格点之间跃迁。随后写出一个几率幅运动方程,并利用拉普拉斯变换,研究$s\to 0$时,$V_n(s)$的渐进性为,其中$V_n(s)$代表从第$n$个格点出发的所有虚跃迁过程的总和。

当$V_{jk}$为短程相互作用,即衰减快于$1/r^3$时,低浓度下$V_n(s\to 0) \to 0$,即基态严格局域化,不存在扩散;对于长程相互作用,如$1/r^3$,虽然$V_n(s\to 0) \neq 0$,但增长极为缓慢,也预示了扩散速率随距离指数衰减。

这篇论文虽然提出了一个非常简化的理论模型,但对于理解和研究一些实际的量子扩散现象具有重要的指导意义。

首先是自旋扩散,$Anderson$最初的动机之一就是研究参杂硅中的自旋扩散行为。在参杂材料中,自旋往往被捕获在居于的缺陷或杂质的位置上,扩散过程发生在这些局域中心之间,本文提出的模型可以用于描述该现象,了解自旋扩散的极限行为。其次,在无序半导体中存在大量无定形区域和缺陷,电子在这些区域中的迁移行为类似于本文所描述的局域态之间的虚拟跃迁过程,该模型也有助于理解缺陷密度、耦合强度等对输运性质的影响。

综上,该文章建立了描述局域缺陷系统中量子扩散的简单模型,证明了在弱耦合极限下必然存在严格的波函数局域化,为无扩散态提供了重要的例证。同时也为研究有扩散态时的输运机制指明了方向,对于理解固体缺陷等各类无序介质中的量子扩散现象提供了重要的物理图像和理论工具。

\textbf{Problem II.2} How the transmittance and reflection coefficients T and R depend on $t_i$ or $r_i$ in Eq. (94)? Discuss the effect of the relative phases.

\begin{proof}
在量子力学理论框架下,不可以对概率直接相乘,而要将其概率幅相乘,取模得到其概率。即
\[
	t = t_1t_2 + t_1r_2r_1t_2 + t_1r_2r_1r_2r_1t_2 + \cdots = \frac{t_1t_2}{1-r_1r_2}.
\]

从而可以计算透射系数:
\[
	T = \vert t \vert^2 = \frac{\vert t_1 \vert^2 \vert t_2 \vert^2}{\vert 1 - r_1r_2 \vert^2}.
\]

类似的,可以计算反射几率幅为:
\[
	r = r_1 + t_1r_2t_1 + t_1r_2r_1r_2t_1 + \cdots = r_1 + \frac{t_1^2r_2}{1-r_1r_2} = \frac{r_1+(t_1^2-r_1^2)r_2}{1-r_1r_2}.
\]

从而计算反射系数为:
\[
	R = \vert r \vert^2 = \frac{\vert r_1+(t_1^2-r_1^2)r_2 \vert^2}{\vert 1 - r_1r_2 \vert^2}.
\]

也可以直接利用透射系数与反射系数的和为1,计算反射系数:
\[
	R = 1 - T = 1 - \frac{\vert t_1 \vert^2 \vert t_2 \vert^2}{\vert 1 - r_1r_2 \vert^2}.
\]

考虑
\[
	t_1 = \vert t_1 \vert e^{i\theta_1}; \ t_2 = \vert t_2 \vert e^{i\theta_2}; \ r_1 = \vert r_1 \vert e^{i\varphi_1}; \ r_2 = \vert r_2 \vert e^{i\varphi_2}.
\]

可以计算透射系数为:
\[
	T = \frac{\vert t_1 \vert^2 \vert t_2 \vert^2}{[1 - \vert r_1 \vert \vert r_2 \vert e^{i(\varphi_1+\varphi_2)}][1 - \vert r_1 \vert \vert r_2 \vert e^{-i(\varphi_1+\varphi_2)}]} = \frac{\vert t_1 \vert^2 \vert t_2 \vert^2}{1 + \vert r_1 \vert^2 \vert r_2 \vert^2 - 2\vert r_1 \vert \vert r_2 \vert \cos(\varphi_1+\varphi_2)}.
\]

反射系数为:
\[
	R = 1 - T = 1 - \frac{\vert t_1 \vert^2 \vert t_2 \vert^2}{1 + \vert r_1 \vert^2 \vert r_2 \vert^2 - 2\vert r_1 \vert \vert r_2 \vert \cos(\varphi_1+\varphi_2)}.
\]

可以看出,透射系数和反射系数与$r_1$和$r_2$的相位和相关。当$\varphi_1+\varphi_2 = 2n\pi, \ n=0,\pm1,\pm2,\cdots$时,$\cos(\varphi_1+\varphi_2) = 1$,此时透射系数最大,反射系数最小;当$\varphi_1+\varphi_2 = (2n+1)\pi, \ n=0,\pm1,\pm2,\cdots$时,$\cos(\varphi_1+\varphi_2) = -1$,此时透射系数最小,反射系数最大。
\end{proof}


\end{document}