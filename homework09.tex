\documentclass[reqno,a4paper,12pt]{amsart}

\usepackage{amsmath,amssymb,amsthm,geometry,xcolor,soul,graphicx}
\usepackage{titlesec}
\usepackage{enumerate}
\usepackage{lipsum}
\usepackage{listings}
%\RequirePackage[most]{tcolorbox}
\usepackage{braket}
\allowdisplaybreaks[4] %align公式跨页
\usepackage{xeCJK}
\setCJKmainfont[AutoFakeBold = true]{Kai}
\geometry{left=0.7in, right=0.7in, top=1in, bottom=1in}

\renewcommand{\baselinestretch}{1.3}

\title{介观物理第九次作业}
\author{董建宇 ~~ 202328000807038}

\begin{document}

\maketitle
\titleformat{\section}[hang]{\small}{\thesection}{0.8em}{}{}
\titleformat{\subsection}[hang]{\small}{\thesubsection}{0.8em}{}{}

\textbf{Problem III.3} Order of magnitude estimation: For a Fermi energy of $E_F$ and $N$ electrons in the metallic ring, the spacing between the bands will be on the order of $E_F/N$. Suppose that $N = 10^4$, estimate the level spacing of a typical ladder and the corresponding temperature scale. Then find the physical condition under which the persistent current can be observed in the presence of inelastic scattering. 

Hint: The width of the Fermi tail should be small compared to the level spacing in Fig. 17.

\begin{proof}
为了进行数量级估计,可以取费米能为$E_F = 5eV$,则能隙大小约为:
\[
	E_{gap} \approx \frac{E_F}{N} = 5\times 10^{-4} eV.
\]

对应温度为:
\[
	T = \frac{E_{gap}}{k_B} = 5.80K.
\]

即温度高于$5.80K$,且圆环尺寸为介观尺度,约小于$10^{-6}m$,可以观察到非弹性散射存在条件下的持续的电流。
\end{proof}


\textbf{Problem III.5} Check Eqs. (193) with the unitary condition $SS^\dagger = \mathbb{I}$.
\begin{align*}
	&T_i = \sum_j T_{ij}, \ \ R_i = \sum_j R_{ij}; \tag{193a} \\
	&\sum_i T_i = \sum_i (1-R_i), \ \ \sum_i T_i' = \sum_i (1-R_i'); \tag{193b} \\
	&R_i' + T_i = 1, \ \ R_i + T_i' = 1; \tag{193c}
\end{align*}

\begin{proof}
$S$矩阵为:
\[
	S = \left( \begin{matrix}
		r & t' \\
		t & r'
	\end{matrix} \right).
\]

其中$r, r'; \ t, t'$都是$N\times N$矩阵。则可以计算:
\[
	SS^\dagger = \left( \begin{matrix}
		r & t' \\
		t & r'
	\end{matrix} \right) \left( \begin{matrix}
		r^\dagger & t^\dagger \\
		t'^\dagger & r'^\dagger
	\end{matrix} \right) = \left( \begin{matrix}
		r r^\dagger + t't'^\dagger & rt^\dagger + t'r'^\dagger \\
		tr^\dagger + r't'^\dagger & tt^\dagger + r'r'^\dagger
	\end{matrix} \right) = \left( \begin{matrix}
		\mathbb{I}_N & 0 \\
		0 & \mathbb{I}_N
	\end{matrix} \right).
\]

其中$\mathbb{I}_N$是$N\times N$的单位矩阵。当具有时间反演对称性时,$S$是一个对称矩阵,即$S = S^T$,从而有
\[
	r = r^T; \ \ r' = r'^T; \ \ t' = t^T; \ \ t = t'^T.
\]

即矩阵元满足如下关系;
\[
	r_{ij} = r_{ji}; \ \ r'_{ij} = r'_{ji}.
\]

根据$rr^\dagger + t't'^\dagger = \mathbb{I}_N$,从而可以计算:
\[
	\sum_m r_{im}r_{im}^* + \sum_n {t'}_{in}{t'}_{in}^* = R_i + T'_i = 1.
\]

类似的,根据$tt^\dagger + r'r'^\dagger = \mathbb{I}_N$,从而可以计算:
\[
	\sum_m t_{im}t_{im}^* + \sum_n {r'}_{in}{r'}_{in}^* = T_i + R'_i = 1.
\]

即验证了式(193c)。

利用$t' = t^T$,则有$t'^\dagger = t^*$,从而有:
\[
	r r^\dagger + t^Tt^* = \mathbb{I}_N.
\]

两侧取迹,则有:
\[
	\sum_i \sum_j R_{ij} + \sum_m \sum_n T_{nm} = \sum_i R_i + \sum_n T_n = \sum_k 1. 
\]

所有求和指标均从$1$求和至$N$。其中由于求和项有限,可以交换$m$和$n$的求和顺序,移项可将方程重新写为:
\[
	\sum_i T_i = \sum_i (1-R_i).
\]

类似的,利用$t = t'^T$,则有$t^\dagger = t'^*$,从而有:
\[
	r' r'^\dagger + t'^Tt'^* = \mathbb{I}_N.
\]

同样的,两侧取迹,交换求和顺序,移项可以得到:
\[
	\sum_i T_i' = \sum_i (1-R_i').
\]

即验证了方程(193b)。
\end{proof}




\end{document}